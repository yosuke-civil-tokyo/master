\chapter*{謝辞}
\addcontentsline{toc}{chapter}{謝辞}
私が交通・都市・国土学研究室に配属され,研究を始めてから多くの方々にお世話になりました.本研究はそうした方々の助力なしでは決して成り立たなかったものです.せめてもの感謝の意を,謝辞で示したいと思います.

初めに,指導教官である浦田淳司講師には配属されてからの約1年間,常に熱心な指導をしていただきました.興味分野が研究室のプロジェクトに完全には一致していない私の研究題目を考えるところから,研究を進める手順や道筋,文章や発表といった伝え方のスキルマで,常に研究を良いものにするための教えを説いて下さいました.研究だけでなく,研究室内で分野知識を学ぶ手助けや,他分野へと知見を広げる機会を頂き,学習面で大いに成長する環境を整えていただきました.姿勢の面でも,先生の常に問題を整理し,実直に向き合う姿勢を始め多くの学びをさせていただきました.心より感謝を申し上げます.

研究室の羽藤英二教授は,週報やゼミをはじめ,様々な機会に研究の方針についての話をしていただきました.先生が常に楽しみながら研究の話をして下さったからこそ,私も自分の興味や楽しみを見失わずに研究を続けることができました.学部3年の講義で,都市・交通分野の学習・研究を行うきっかけを下さったことも含め,非常にお世話になっております.大変感謝しています.

研究室の方々にはとてもお世話になりました.北原麻理奈先生,芝原貴史先生,萩原拓也先生は,ゼミをはじめ研究室での学びを大いに支えて下さると共に,私が現在研究している分野から少し離れた専門的な研究のお話をしていただきました.秘書の黒田由佳さん,森容子さん,高瀬理恵さん,中井靖子さんは謝金をはじめ,書類処理で大変お世話になりました.先輩方は,頼れる知識と経験で,いつも私を支えて下さいました.博士課程の小林里瑳さんは,春学期のゼミにて,分野の知識が全くない私たちB4が多くを学べるようにゼミの進行,要点の解説を行なって下さいました.M2の方々には,修士論文をはじめ,多くの話をしていただきました.新井拓朗さんは身を削りながら東北での過酷な研究生活の話をしてくださり,小川瑞希さんは強化学習のバンディットの話に加え,ぬいぐるみを研究室に買ってくださりました.小関玲奈さんからは,研究を通して地元への愛着を学ばせていただき,Shen Binchangさんの研究からは機械学習の交通分野での可能性を大いに感じさせていただきました.須賀拓実さんの日常の姿勢からは,学ぶべき部分が多くありました.黛風雅さんがいつも声をかけてくださったおかげで研究室にも行きやすくなり,渡邉葵さんがしてくださる研究や関連知識の話は面白いものばかりでした.卒論提出日の味噌汁は心に沁みました.M1の方々は,主に春学期ゼミで多くのことを教えて下さいました.小島元太朗さんは夏の学校で庶務の作業を指揮してくださり,鈴木大樹さんは忙しい春学期にゼミでの課題だけでなく,均衡配分コードの一部解説をしてくださりました.月田光さんはRLの説明をしてくださり,そのおかげで夏の学校でRLを用いることができました.前田歩美さんはEMアルゴリズムの説明をしてくださり,夏の学校では作業の姿勢をスラック上で実践して示してくださりました.増田慧樹さんは講義を受けながら研究を熱心に行っており,研究を行うにあたり,とても励まされました.その上,論文内の図の作成を手助けしていただきました.ありがとうございます.研究室の同期は,研究を行う上で心強い存在でした.近藤愛子さんはゼミを共に行うことも多く,研究やプレゼンなどの相談に多く乗ってもらいました.増橋佳菜さんは因果分析とネットワークデータ,行動データを用いてとても面白い研究をしており,分野への興味を強めてくれました.村橋拓真くんが自分のペースで問題に向き合い,何より研究を楽しんでいる点は羨ましくも思い,研究への向き合い方を見直させてくれました.研究室の皆様,楽しく学びが多い研究室での生活を送らせていただき,本当にありがとうございます.

忙しい中,小林さんと渡邉さん,鈴木さんは論文の校正をとても詳細に,短い期間で行なっていただきました.皆様のおかげで,最後論文をより良いものに仕上げることができました.感謝申し上げます.

また,いつも楽しげなやりとりをしてくれた谷口史花さんをはじめとして,日常生活を支えてくれた友人らにはこの場で感謝を述べたいと思います.

最後に,いつも気の休まる帰る場所を用意してくれている両親や叔父に感謝を申し上げ,謝辞の結びとさせていただきます.

{\medskip}
\hfill 2022年2月 研究室にて

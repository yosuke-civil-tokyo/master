\chapter{序論} 
\label{1}
\section{研究の背景}\label{1.1}
交通需要の予測のために,近年ではActivity-based Model(ABM)の研究と開発が行われている.
ABMは交通需要を活動の派生需要として捉えるモデルであり,交通需要を集計的に,かつ複数の計算プロセスを複数のモデルに分けて交通需要を予測する四段階推定法に代わり,個人単位の活動スケジュールの予測を一括のモデルで予測する.
これによりABMは,計算プロセス間の整合性を図つつ,より詳細かつ複数の単位での集計分析を可能にしている.

一方で,多数の変数を扱うABMは,その計算を行うために複雑なモデル構造を要する.
ABMでは変数間の計算関係(モデル構造と呼ぶ)により各個人の行動選択構造を記述し,スケジュールの生成を行なっている.
ここで,ABM内の変数は,主に各個人の属性変数に加えてスケジュールを記述するための変数を指す.
ABMのモデル構造は,その精度や利用性に大きく影響を持つことが知られており,数々のモデル構造が提案されている.
複数のモデル構造が提案される背景には,ヒューリスティクスや統計の知識を用いることで,個人の行動選択を現実に近いものとし,活動生成の精度を向上させる目的がある.
加えて,目的に応じてより詳細な変数を扱う目的や,モデルのデータ同化に用いるデータ種類に応じて,適したモデル構造が提案されてきた.



\section{研究の目的}\label{1.2}


\section{構成}\label{1.3}
